\documentclass{article}
\usepackage[utf8]{inputenc}
\usepackage{amsmath, amssymb}
\usepackage{tabularx}

\def\Ltwo{\text{L}^2}
\newcommand{\norm}[1]{\left\|#1\right\|}
\newcommand{\laplacian}{\Delta}

\begin{document}
\title{Kompleksowa Analiza Fouriera w Zagadnieniach Dyfuzji}
\author{Autor Syntezujący}
\date{\today}

\maketitle

\begin{abstract}
Praca prezentuje zwięzłą analizę rozkładu funkcji $f(t)$ za pomocą szeregów Fouriera w kontekście równania ciepła na prostokącie. Wykazano, że błąd aproksymacji jest ściśle związany z wartościami własnymi operatora Laplace'a, a rozwiązanie w szereg jest uogólnieniem teorii Fouriera.
\end{abstract}

\section{Wstęp do Przestrzeni Hilberta}

Rozważmy funkcję całkowalną z kwadratem. Szereg Fouriera jest zdefiniowany następująco:
\[ f = \sum c_n \Phi_n \]
Równość Parsevala stanowi podstawowy element teorii. Mówi ona, że norma funkcji w przestrzeni Hilberta jest równa sumie kwadratów norm współczynników:
\begin{equation}
    \norm{f}_{L^2}^2 = \sum |c_n|^2
\end{equation}
Jeśli funkcja jest ciągła, współczynniki maleją jak $O(1/n)$, co jest kluczowe dla szybkości zbieżności.

\section{Operator Laplace'a i Separacja}

W problemach wielowymiarowych poszukujemy rozwiązania w postaci iloczynu. Prowadzi to do zagadnienia własnego:
$$ -\laplacian u = \lambda u $$
Dla obszaru prostokątnego $R$, wartości własne $\lambda_{m, n}$ są sumą składowych $\mu_m$ i $\nu_n$:
$$ \lambda_{m, n} = \left( \frac{m\pi}{a} \right)^2 + \left( \frac{n\pi}{b} \right)^2 $$
Funkcje te tworzą pełny zbiór ortogonalny w przestrzeni $\Ltwo(R)$.

\section{Zastosowanie: Równanie Ciepła}

Zgodnie z zasadą superpozycji, rozwiązanie problemu początkowo-brzegowego dla równania ciepła wyraża się wzorem:
$$ u(x, t) = \sum_{n=1}^\infty B_n \Phi_n(x) e^{-k \lambda_n t} $$

Poniższa tabela łączy analizę błędu (z ogólnej teorii) z konkretnymi czasami zaniku dla ustalonych wartości własnych:

\begin{center}
\begin{tabular}{|c|c|c|c|}
\hline
$N$ (wyrazy) & $\lambda_N$ & $\norm{\epsilon_N}$ (błąd) & Czas zaniku $T$ \\
\hline
10 & 19.739 & $3.1 \times 10^{-2}$ & 0.05 s \\
100 & 49.348 & $5.9 \times 10^{-3}$ & 0.85 s \\
1000 & 78.957 & $1.1 \times 10^{-4}$ & 98.7 s \\
\hline
\end{tabular}
\end{center}

Metoda Fouriera pozwala na efektywne modelowanie, a analiza numeryczna potwierdza, że dla dużych $t$ dominują małe wartości własne.

\end{document}