\documentclass{article}
\usepackage[utf8]{inputenc}
\usepackage{amsmath, amssymb}
\usepackage{tabularx}

\newcommand{\macierz}[1]{\mathbf{#1}}

\begin{document}
\title{Metody Numeryczne: Rozkład SVD}
\maketitle

\begin{abstract}
Artykuł opisuje algorytm rozkładu według wartości osobliwych (SVD) dla macierzy prostokątnych. Jest to fundamentalne twierdzenie w algebrze liniowej.
\end{abstract}

\section{Wprowadzenie}
Niech $A$ będzie macierzą o wymiarach $m \times n$. Rozkład SVD pozwala zapisać ją jako iloczyn trzech macierzy:
\[ A = U \Sigma V^T \]
Gdzie $U$ i $V$ są ortogonalne, a $\Sigma$ jest diagonalna.

\section{Właściwości}
Wartości na przekątnej $\Sigma$ nazywamy wartościami osobliwymi.
Norma Frobeniusa macierzy $A$ może być obliczona jako:
\begin{equation}
    \|A\|_F = \sqrt{\sum_{i=1}^{\min(m,n)} \sigma_i^2}
\end{equation}

Jest to wynik przydatny przy kompresji obrazów.

\section{Złożoność obliczeniowa}
Poniższa tabela przedstawia czas działania algorytmu Golub-Kahan.

\begin{center}
\begin{tabular}{|l|r|}
\hline
Rozmiar macierzy & Czas (ms) \\
\hline
$100 \times 100$ & 12 \\
$1000 \times 1000$ & 450 \\
\hline
\end{tabular}
\end{center}

\end{document}