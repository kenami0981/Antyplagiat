\documentclass{article}
\usepackage[utf8]{inputenc}
\usepackage[T1]{fontenc}
\usepackage{amsmath, amssymb}

\newcommand{\partiald}[2]{\frac{\partial #1}{\partial #2}}
\newcommand{\partialt}[1]{\frac{\partial #1}{\partial t}}
\newcommand{\laplacian}{\Delta}

\begin{document}
\title{Ewolucja Ciepła i Wartości Własne}
\author{Ekspert Od Ciepła}
\date{Styczeń 2026}

\maketitle

\begin{abstract}
Artykuł omawia zastosowanie metody rozdzielania zmiennych do równania przewodnictwa cieplnego na obszarze prostokątnym. Przedstawiono analityczne rozwiązanie problemu oraz dyskusję na temat ortogonalności funkcji własnych, które determinują szybkość zaniku temperatury w czasie.
\end{abstract}

\section{Sformułowanie Problemu}

Analizie poddano obszar z jednorodnymi warunkami brzegowymi Dirichleta $u = 0$ na brzegu. Równanie dyfuzji prowadzi do poszukiwania rozwiązań w postaci:
$$ \partialt{u} = k \laplacian u $$
gdzie operatorem przestrzennym jest:
$$ \laplacian u = \partiald{^2 u}{x^2} + \partiald{^2 u}{y^2} $$

Stosując metodę separacji zmiennych $u(x,y,t) = \Phi(x,y)T(t)$, otrzymujemy zagadnienie własne dla części przestrzennej:
$$ -\laplacian \Phi = \lambda \Phi $$
oraz równanie czasowe $T'(t) + k\lambda T(t) = 0$.

\section{Wartości Własne i Rozwiązanie}

Wartości własne $\lambda_{m, n}$ są dane wzorem znanym z analizy na prostokącie $R=[0, a] \times [0, b]$:
$$ \lambda_{m, n} = \left( \frac{m\pi}{a} \right)^2 + \left( \frac{n\pi}{b} \right)^2 $$
Odpowiadające im funkcje własne $\Phi_{m, n}$ są iloczynami sinusów. Zgodnie z zasadą superpozycji, rozwiązanie ogólne wyraża się wzorem zawierającym czynnik wykładniczy:

$$ u(x, y, t) = \sum_{m,n} C_{m,n} \sin\left(\frac{m\pi x}{a}\right) \sin\left(\frac{n\pi y}{b}\right) e^{-k \lambda_{m,n} t} $$

\section{Analiza Numeryczna Zaniku}

Obecność czynnika $e^{-\lambda t}$ gwarantuje bardzo szybką zbieżność. Poniższa tabela prezentuje ewolucję temperatury (korzystając z przykładowych wartości własnych dla $a=b=1$):

\begin{center}
\begin{tabular}{|c|c|c|c|}
\hline
Czas $t$ & $\lambda_{1,1} \approx 19.7$ & $\lambda_{2,2} \approx 78.9$ & Amplituda \\
\hline
0.0 & 1.000 & 1.000 & 1.000 \\
0.1 & 0.139 & 0.000 & 0.139 \\
0.5 & 0.000 & 0.000 & 0.000 \\
\hline
\end{tabular}
\end{center}

Zauważalny jest szybki zanik wyższych harmonicznych. Wartości własne są w pełni określone analitycznie, a ich charakterystyka jest kluczowa dla analizy rozkładów temperatur.

\end{document}