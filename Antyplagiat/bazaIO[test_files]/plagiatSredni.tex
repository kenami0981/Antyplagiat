\documentclass{article}
\usepackage[utf8]{inputenc}
\usepackage{amsmath}

\def\R{\mathbb{R}}

\begin{document}
\section{Analiza sygnałów}

W inżynierii często spotykamy się z problemem filtracji. Jednakże, wracając do teorii matematycznej, rozważmy funkcję $f: [0, 2\pi] \to \R$ całkowalną z kwadratem. Jak wiadomo, każdą taką funkcję można przedstawić jako szereg trygonometryczny.

Podstawowym narzędziem jest Równość Parsevala:
\[ \sum_{n=-\infty}^{\infty} |c_n|^2 = \frac{1}{2\pi} \int_{-\pi}^{\pi} |f(x)|^2 dx \]

Jest to nieco inna forma niż w pracach teoretycznych, ale zasada pozostaje ta sama: norma funkcji w przestrzeni Hilberta jest równa sumie kwadratów norm współczynników.

\section{Wyniki symulacji}
Przeprowadziliśmy własne badania dla funkcji $f(x) = x^2$.

\begin{center}
\begin{tabular}{|c|c|}
\hline
Próbki & Błąd \\
\hline
50 & $2.5 \times 10^{-2}$ \\
500 & $4.1 \times 10^{-4}$ \\
\hline
\end{tabular}
\end{center}

Wnioski są zbieżne z teorią - błąd maleje wraz ze wzrostem N.
\end{document}