\documentclass{article}
\usepackage[utf8]{inputenc}
\usepackage[T1]{fontenc}
\usepackage{amsmath, amssymb, amsthm}
\usepackage{graphicx}
\usepackage{tabularx}

% Mieszanka makr z obu plików
\def\reals{\mathbb{R}}
\def\Ltwo{\text{L}^2}
\newcommand{\laplacian}{\Delta}
\newcommand{\norm}[1]{\left\|#1\right\|}

\begin{document}
\title{Analiza Spektralna i Aproksymacja w Obszarze Prostokątnym}
\author{Autor Mieszany}
\date{\today}

\maketitle

\begin{abstract}
W niniejszym dokumencie przedstawiono analizę zagadnienia własnego dla operatora Laplace'a na prostokątnym obszarze $R$ połączoną z analizą błędu w przestrzeni $L^2$. Wykazano, że błąd jest ściśle związany z szybkością, z jaką współczynniki dążą do zera, a funkcje własne tworzą pełny zbiór ortogonalny.
\end{abstract}

\section{Wprowadzenie do Równania Własnego}

Rozważmy funkcję całkowalną z kwadratem. Jak wiadomo, każdą taką funkcję można przedstawić jako szereg, co prowadzi do następującego zagadnienia własnego:
$$ -\laplacian u = \lambda u \quad \text{w } R $$
z warunkami brzegowymi Dirichleta. Operatorem Laplace'a w dwóch wymiarach jest suma drugich pochodnych cząstkowych. Kwestia ta jest gruntownie omówiona w literaturze, a początkowe kroki opierają się na założeniu ortogonalności.

\section{Rozkład i Równość Parsevala}

Poszukujemy rozwiązania $u(x, y)$ w postaci iloczynu funkcji jednowymiarowych. Podstawienie tej formy daje równanie:
$$ -\frac{X''(x)}{X(x)} = \frac{Y''(y)}{Y(y)} + \lambda = \mu $$
Wartości własne $\lambda_{m, n}$ operatora Laplace'a są sumą $\mu_m$ i $\nu_n$.
Jednocześnie, równość Parsevala stanowi podstawowy element teorii. Mówi ona, że norma funkcji w przestrzeni Hilberta jest równa sumie kwadratów norm współczynników:
\begin{equation}
    \norm{f}_{L^2}^2 = \sum_{m=1}^\infty \sum_{n=1}^\infty |c_{m, n}|^2
\end{equation}

\section{Analiza Błędu i Wnioski}

Błąd aproksymacji $\epsilon_N$ po $N$ wyrazach szeregu jest definiowany jako $f$ minus suma częściowa. Poniższa tabela przedstawia wybrane wartości własne oraz błąd aproksymacji:

\begin{center}
\begin{tabular}{|c|c|c|c|}
\hline
$m$ & $n$ & $\lambda_{m, n} / \pi^2$ & $\norm{\epsilon_N}$ \\
\hline
1 & 1 & 2 & $3.1 \times 10^{-2}$ \\
2 & 1 & 5 & $5.9 \times 10^{-3}$ \\
2 & 2 & 8 & $1.1 \times 10^{-4}$ \\
\hline
\end{tabular}
\end{center}

Wnioskujemy, że rozkład w szereg jest uogólnieniem teorii Fouriera, a błąd maleje wraz ze wzrostem liczby funkcji bazowych.

\end{document}