\documentclass{article}
\usepackage[utf8]{inputenc}
\usepackage[T1]{fontenc} % Lepsze wsparcie dla polskich znaków
\usepackage{amsmath, amssymb}
\usepackage{amsthm} % Dla definicji, twierdzeń itp.
\usepackage{graphicx}
\usepackage{tabularx}

% Definicje makr i symboli
\newcommand{\laplacian}{\Delta}
\newcommand{\partiald}[2]{\frac{\partial #1}{\partial #2}}
\def\reals{\mathbb{R}}
\def\complex{\mathbb{C}}
\def\Ltwo{\text{L}^2}

\newtheorem{definition}{Definicja}

\begin{document}
\title{Wartości Własne Operatora Laplace'a w Obszarze Prostokątnym}
\author{Przykładowy Ekspert}
\date{23 listopada 2025}

\maketitle

\begin{abstract}
W niniejszym dokumencie przedstawiono analizę zagadnienia własnego dla operatora Laplace'a na prostokątnym obszarze $R = [0, a] \times [0, b]$ z jednorodnymi warunkami brzegowymi Dirichleta. Przedstawiono analityczne rozwiązanie problemu oraz dyskusję na temat ortogonalności funkcji własnych w przestrzeni $\Ltwo(R)$.
\end{abstract}

\section{Wprowadzenie do Równania Własnego}

Równanie Helmholtza, wynikające z poszukiwania stacjonarnych rozwiązań równania falowego lub dyfuzji, prowadzi do następującego zagadnienia własnego:

$$ -\laplacian u = \lambda u \quad \text{w } R $$

z warunkami brzegowymi Dirichleta $u(x, y) = 0$ na $\partial R$. Operatorem Laplace'a, $\laplacian$, w dwóch wymiarach jest:
$$ \laplacian u = \partiald{^2 u}{x^2} + \partiald{^2 u}{y^2} $$

\section{Metoda Rozdzielania Zmiennych}

Poszukujemy rozwiązania $u(x, y)$ w postaci iloczynu funkcji jednowymiarowych $u(x, y) = X(x)Y(y)$. Podstawienie tej formy do równania własnego daje:

$$ -\left( X''(x) Y(y) + X(x) Y''(y) \right) = \lambda X(x) Y(y) $$

Po podzieleniu przez $X(x)Y(y)$ i przegrupowaniu otrzymujemy separację zmiennych:
\begin{equation}
    -\frac{X''(x)}{X(x)} = \frac{Y''(y)}{Y(y)} + \lambda = \mu
    \label{eq:separacja}
\end{equation}
gdzie $\mu$ jest stałą separacji.

\subsection{Zagadnienia Jednowymiarowe}

Równanie (\ref{eq:separacja}) prowadzi do dwóch niezależnych zagadnień własnych (Sturm-Liouville'a):
\begin{align}
    X''(x) + \mu X(x) &= 0, \quad X(0) = X(a) = 0 \\
    Y''(y) + (\lambda - \mu) Y(y) &= 0, \quad Y(0) = Y(b) = 0
\end{align}

Rozwiązania są znane. Wartości własne $\mu_m$ i funkcje własne $X_m(x)$ dla $x$-owej części to:
$$ \mu_m = \left( \frac{m\pi}{a} \right)^2, \quad X_m(x) = \sin\left(\frac{m\pi x}{a}\right), \quad m \in \mathbb{N} $$
Analogicznie, dla $y$-owej części, stała $(\lambda - \mu)$ musi być wartością własną, którą oznaczamy jako $\nu_n$:
$$ \nu_n = \left( \frac{n\pi}{b} \right)^2, \quad n \in \mathbb{N} $$

\section{Wartości i Funkcje Własne}

Łączna wartość własna $\lambda_{m, n}$ jest sumą $\mu_m$ i $\nu_n$:

\begin{definition}
Wartości własne $\lambda_{m, n}$ operatora Laplace'a na prostokącie $R=[0, a] \times [0, b]$ są dane wzorem:
$$ \lambda_{m, n} = \left( \frac{m\pi}{a} \right)^2 + \left( \frac{n\pi}{b} \right)^2, \quad m, n \in \mathbb{N} $$
\end{definition}

Odpowiadające im funkcje własne $\Phi_{m, n}(x, y)$ są iloczynami funkcji jednowymiarowych:
$$ \Phi_{m, n}(x, y) = \sin\left(\frac{m\pi x}{a}\right) \sin\left(\frac{n\pi y}{b}\right) $$


\section{Ortogonalność i Zbieżność}

Funkcje własne $\Phi_{m, n}$ tworzą pełny zbiór ortogonalny w przestrzeni Hilberta $\Ltwo(R)$. Oznacza to, że dowolną funkcję $f \in \Ltwo(R)$ można rozłożyć w szereg:
$$ f(x, y) = \sum_{m=1}^\infty \sum_{n=1}^\infty c_{m, n} \Phi_{m, n}(x, y) $$

Poniższa tabela przedstawia kilka pierwszych wartości własnych dla $a=1$ i $b=1$.

\begin{center}
\caption{Wybrane wartości własne $\lambda_{m, n}$ dla $a=b=1$}
\begin{tabular}{|c|c|c|c|}
\hline
$m$ & $n$ & $\lambda_{m, n} / \pi^2$ & $\lambda_{m, n}$ (przybliżone) \\
\hline
1 & 1 & 2 & 19.739 \\
2 & 1 & 5 & 49.348 \\
1 & 2 & 5 & 49.348 \\
2 & 2 & 8 & 78.957 \\
\hline
\end{tabular}
\end{center}

\section{Wnioski}

Wartości własne operatora Laplace'a na prostokącie są w pełni określone analitycznie, a ich charakterystyka jest kluczowa dla analizy drgań membran czy rozkładów temperatur. Rozkład w szereg jest uogólnieniem teorii Fouriera.
\end{document}