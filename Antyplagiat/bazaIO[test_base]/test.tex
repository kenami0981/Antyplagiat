\documentclass{article}
\usepackage[utf8]{inputenc}
\usepackage{amsmath, amssymb}
\usepackage{graphicx}
\usepackage{tabularx}

% Definicje makr i symboli
\def\R{\mathbb{R}}
\def\N{\mathbb{N}}
\newcommand{\norm}[1]{\left\|#1\right\|}
\newcommand{\wektor}[1]{\mathbf{#1}}

\begin{document}
\title{Analiza Fouriera na Przestrzeniach Hilberta}
\author{Testowy Autor}
\date{\today}

\maketitle

\begin{abstract}
Praca prezentuje zwięzłą analizę rozkładu funkcji $f(t)$ za pomocą szeregów Fouriera. Koncentrujemy się na warunku zbieżności i analizie błędu w przestrzeni $L^2$. Wykazano, że błąd jest ściśle związany z szybkością, z jaką współczynniki $c_n$ dążą do zera.
\end{abstract}

\section{Wprowadzenie}

Rozważmy funkcję $f: [0, 2\pi] \to \R$ całkowalną z kwadratem. Jak wiadomo, każdą taką funkcję można przedstawić jako szereg trygonometryczny. Kwestia ta jest gruntownie omówiona w literaturze \cite{KsięgaAnaliza}. Początkowe kroki opierają się na założeniu ortogonalności.

\section{Szereg Fouriera i Przestrzenie $L^2$}

Szereg Fouriera dla $f(t)$ jest zdefiniowany następująco:
\[ f(t) = a_0 + \sum_{n=1}^\infty (a_n \cos(nt) + b_n \sin(nt)) \]
gdzie współczynniki $a_n$ i $b_n$ obliczane są przez standardowe całki. Równość Parsevala stanowi podstawowy element teorii. Mówi ona, że norma funkcji w przestrzeni Hilberta jest równa sumie kwadratów norm współczynników. Wyraża się to wzorem:

\begin{align}
    \norm{f}_{L^2}^2 &= \frac{1}{\pi} \int_0^{2\pi} |f(t)|^2 dt \label{eq:norma}\\
    &= 2a_0^2 + \sum_{n=1}^\infty (a_n^2 + b_n^2) \label{eq:parseval}
\end{align}

W praktyce często używamy notacji zespolonej dla współczynników $c_n$.

\begin{gather}
    c_n = \frac{1}{2\pi} \int_0^{2\pi} f(t) e^{-int} dt \\
    f(t) = \sum_{n=-\infty}^\infty c_n e^{int} \label{eq:zesp}
\end{gather}

Jeśli $f(t)$ jest ciągła, a jej pochodna $f'(t)$ jest kawałkami ciągła, to współczynniki $c_n$ maleją jak $O(1/n)$. Jest to kluczowe dla szybkości zbieżności.

\section{Analiza Błędu}

Błąd aproksymacji $\epsilon_N$ po $N$ wyrazach szeregu jest definiowany jako $f(t)$ minus suma częściowa $S_N(t)$.
Chcemy, aby $\norm{\epsilon_N}_{L^2} \to 0$ gdy $N \to \infty$. Oczywiście, $\frac{\norm{\wektor{v}}}{\norm{\wektor{u}}}$ to stosunek norm.

\subsection{Wyniki Numeryczne}

Poniższa tabela pokazuje przykładowe czasy obliczeń. Czas $T_N$ zależy od $N$: $T_N \approx N^2$.

\begin{center}
\begin{tabular}{|c|c|c|}
\hline
$N$ & $\norm{\epsilon_N}$ & Czas (s) \\
\hline
10 & $3.1 \times 10^{-2}$ & 0.05 \\
100 & $5.9 \times 10^{-3}$ & 0.85 \\
1000 & $1.1 \times 10^{-4}$ & 98.7 \\
\hline
\end{tabular}
\end{center}

Obserwujemy, że na zbiorze $\R$ analiza jest nieco bardziej skomplikowana.

\section{Wnioski}

Praca potwierdza, że rozkład Fouriera jest efektywnym narzędziem do analizy funkcji $L^2$. Pokazuje to zarówno teoria (Równanie \ref{eq:parseval}), jak i prosta analiza numeryczna.
\end{document}