\documentclass{article}
\usepackage[utf8]{inputenc}
\usepackage[T1]{fontenc}
\usepackage{amsmath, amssymb}
\usepackage{amsthm}
\usepackage{graphicx}
\usepackage{tabularx}

% Definicje makr i symboli
\newcommand{\partialt}[1]{\frac{\partial #1}{\partial t}}
\newcommand{\partialxx}[1]{\frac{\partial^2 #1}{\partial x^2}}
\def\reals{\mathbb{R}}
\def\natural{\mathbb{N}}
\newcommand{\funcSpace}{L^2([0, L])}

\newtheorem{theorem}{Twierdzenie}

\begin{document}
\title{Analityczne Rozwiązanie Równania Ciepła w Pręcie 1D}
\author{Autor Symulowany}
\date{12 stycznia 2026}

\maketitle

\begin{abstract}
Niniejszy artykuł omawia zastosowanie metody rozdzielania zmiennych do rozwiązania jednowymiarowego równania przewodnictwa cieplnego.
Analizie poddano pręt o skończonej długości z jednorodnymi warunkami brzegowymi Dirichleta.
Przedstawiono wyprowadzenie szeregu Fouriera jako rozwiązania ogólnego oraz numeryczną ilustrację zaniku temperatury w czasie.
\end{abstract}

\section{Sformułowanie Problemu}

Rozważmy jednorodny pręt o długości $L$, izolowany na powierzchni bocznej. Rozkład temperatury $u(x, t)$ opisuje klasyczne paraboliczne równanie różniczkowe cząstkowe:

$$ \partialt{u} = k \partialxx{u} \quad \text{dla } x \in (0, L), \ t > 0 $$

gdzie $k > 0$ jest współczynnikiem dyfuzyjności cieplnej. Przyjmujemy następujące warunki brzegowe (końce pręta utrzymywane w temperaturze zero):
\begin{equation}
    u(0, t) = 0, \quad u(L, t) = 0
    \label{eq:brzegowe}
\end{equation}
oraz warunek początkowy $u(x, 0) = f(x)$, gdzie $f(x)$ jest zadaną funkcją rozkładu temperatury.

\section{Separacja Zmiennych}

Zakładamy rozwiązanie w postaci iloczynu funkcji przestrzennej i czasowej: $u(x, t) = X(x)T(t)$.
Podstawiając do równania wyjściowego i dzieląc przez $k X(x)T(t)$, otrzymujemy:

\begin{equation}
    \frac{1}{k} \frac{T'(t)}{T(t)} = \frac{X''(x)}{X(x)} = -\lambda
\end{equation}

Stała separacji $-\lambda$ musi być ujemna, aby rozwiązanie nie rosło nieograniczenie w czasie. Prowadzi to do zagadnienia własnego dla zmiennej przestrzennej:
$$ X''(x) + \lambda X(x) = 0 $$
z warunkami $X(0) = X(L) = 0$.

\subsection{Wartości Własne}

Rozwiązaniami powyższego równania są funkcje trygonometryczne. Uwzględniając warunki brzegowe, otrzymujemy dyskretny zbiór wartości własnych:
$$ \lambda_n = \left( \frac{n\pi}{L} \right)^2, \quad n \in \natural $$
oraz odpowiadające im funkcje własne $X_n(x) = \sin\left(\frac{n\pi x}{L}\right)$.

Część czasowa rozwiązania przyjmuje postać wykładniczą:
$$ T_n(t) = C_n e^{-k \lambda_n t} $$

\section{Rozwiązanie Ogólne i Zbieżność}

Zgodnie z zasadą superpozycji, rozwiązanie ogólne jest sumą nieskończoną rozwiązań szczegółowych:

\begin{theorem}
Rozwiązanie problemu początkowo-brzegowego dla równania ciepła wyraża się wzorem:
$$ u(x, t) = \sum_{n=1}^\infty B_n \sin\left(\frac{n\pi x}{L}\right) e^{-k \left(\frac{n\pi}{L}\right)^2 t} $$
gdzie współczynniki $B_n$ są współczynnikami Fouriera funkcji początkowej $f(x)$.
\end{theorem}

Obecność czynnika wykładniczego $e^{-\lambda_n t}$ gwarantuje bardzo szybką zbieżność szeregu dla $t > 0$, co wygładza wszelkie nieciągłości warunku początkowego.

\section{Analiza Numeryczna}

Poniższa tabela prezentuje ewolucję temperatury w środku pręta ($x = L/2$) dla $L=\pi$ i $k=1$, przy założeniu, że w chwili $t=0$ temperatura wynosiła stałą wartość (przybliżenie pierwszymi wyrazami szeregu).

\begin{center}
\caption{Zanik temperatury $u(L/2, t)$ w funkcji czasu}
\begin{tabular}{|c|c|c|c|}
\hline
Czas $t$ & Wyraz $n=1$ & Wyraz $n=3$ & Suma (przybliżona) \\
\hline
0.0 & 1.273 & -0.424 & 0.849 \\
0.1 & 1.152 & -0.174 & 0.978 \\
0.5 & 0.773 & -0.005 & 0.768 \\
1.0 & 0.468 & 0.000 & 0.468 \\
2.0 & 0.172 & 0.000 & 0.172 \\
\hline
\end{tabular}
\end{center}

Zauważalny jest szybki zanik wyższych harmonicznych (dla $n=3$ tłumienie jest 9 razy szybsze niż dla $n=1$).

\section{Podsumowanie}

Metoda Fouriera pozwala na efektywne modelowanie dyfuzji ciepła. Kluczowym wnioskiem jest obserwacja, że po krótkim czasie rozkład temperatury zdominowany jest przez pierwszą funkcję własną, niezależnie od skomplikowania warunku początkowego.

\end{document}